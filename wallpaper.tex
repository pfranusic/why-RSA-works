%%%% why-RSA-works/wallpaper.tex
%%%% Copyright 2012 Peter Franusic.
%%%% All rights reserved.
%%%%

%%%% The goal here is to introduce the Carmichael identity.
%%%% We look at a modex table where a visible pattern is readily apparent.
%%%% We define the Carmichael function value and set forth the wallpaper theorem
%%%% and the Carmichael identity.

%% Introduce the modex-33 table and point out the wallpaper pattern.
We now consider a larger exponentiation table.
Table \ref{modex-33} specifies exponential products $m^a$ in the ring $\mathcal{R}_{33}$.
The table is small enough to fit on a page 
yet big enough for us to visually perceive a \emph{wallpaper} pattern.
There appear to be three identical strips of wallpaper side by side.
Each strip is 10 columns wide and runs from top to bottom.

%% modex-33 table
\begin{table}[!h]
\hspace{-9ex}
  %%%% why-RSA-works/modex-33.tex
%%%% Copyright 2012 Peter Franusic.
%%%% All rights reserved.
%%%%
\begin{footnotesize}
\begin{tabular}
    {c@{ }c@{ }c@{ }c@{ }c@{ }c@{ }c@{ }c@{ }c@{ }c@{ }c@{ }c@{ }c@{ }c@{ }c@{ }c@{ }c@{ }c@{ }c@{ }c@{ }c@{ }c@{ }c@{ }c@{ }c@{ }c@{ }c@{ }c@{ }c@{ }c@{ }c@{ }c@{ }c@{ }c@{ }c}
        & \phantom{X}
             &  0 &  1 &  2 &  3 &  4 &  5 &  6 &  7 &  8 &  9 & 10 & 11 & 12 & 13 & 14 & 15 & 16 & 17 & 18 & 19 & 20 & 21 & 22 & 23 & 24 & 25 & 26 & 27 & 28 & 29 & 30 & 31 & 32 \\
        &    & \phantom{99} &    &    &    &    &    &    &    &    &    &    &    &    &    &    &    &    &    &    &    &    &    &    &    &    &    &    &    &    &    &    &    &    \\
    0   &    &  1 &  0 &  0 &  0 &  0 &  0 &  0 &  0 &  0 &  0 &  0 &  0 &  0 &  0 &  0 &  0 &  0 &  0 &  0 &  0 &  0 &  0 &  0 &  0 &  0 &  0 &  0 &  0 &  0 &  0 &  0 &  0 &  0 \\
    1   &    &  1 &  1 &  1 &  1 &  1 &  1 &  1 &  1 &  1 &  1 &  1 &  1 &  1 &  1 &  1 &  1 &  1 &  1 &  1 &  1 &  1 &  1 &  1 &  1 &  1 &  1 &  1 &  1 &  1 &  1 &  1 &  1 &  1 \\
    2   &    &  1 &  2 &  4 &  8 & 16 & 32 & 31 & 29 & 25 & 17 &  1 &  2 &  4 &  8 & 16 & 32 & 31 & 29 & 25 & 17 &  1 &  2 &  4 &  8 & 16 & 32 & 31 & 29 & 25 & 17 &  1 &  2 &  4 \\
    3   &    &  1 &  3 &  9 & 27 & 15 & 12 &  3 &  9 & 27 & 15 & 12 &  3 &  9 & 27 & 15 & 12 &  3 &  9 & 27 & 15 & 12 &  3 &  9 & 27 & 15 & 12 &  3 &  9 & 27 & 15 & 12 &  3 &  9 \\
    4   &    &  1 &  4 & 16 & 31 & 25 &  1 &  4 & 16 & 31 & 25 &  1 &  4 & 16 & 31 & 25 &  1 &  4 & 16 & 31 & 25 &  1 &  4 & 16 & 31 & 25 &  1 &  4 & 16 & 31 & 25 &  1 &  4 & 16 \\
    5   &    &  1 &  5 & 25 & 26 & 31 & 23 & 16 & 14 &  4 & 20 &  1 &  5 & 25 & 26 & 31 & 23 & 16 & 14 &  4 & 20 &  1 &  5 & 25 & 26 & 31 & 23 & 16 & 14 &  4 & 20 &  1 &  5 & 25 \\
    6   &    &  1 &  6 &  3 & 18 &  9 & 21 & 27 & 30 & 15 & 24 & 12 &  6 &  3 & 18 &  9 & 21 & 27 & 30 & 15 & 24 & 12 &  6 &  3 & 18 &  9 & 21 & 27 & 30 & 15 & 24 & 12 &  6 &  3 \\
    7   &    &  1 &  7 & 16 & 13 & 25 & 10 &  4 & 28 & 31 & 19 &  1 &  7 & 16 & 13 & 25 & 10 &  4 & 28 & 31 & 19 &  1 &  7 & 16 & 13 & 25 & 10 &  4 & 28 & 31 & 19 &  1 &  7 & 16 \\
    8   &    &  1 &  8 & 31 & 17 &  4 & 32 & 25 &  2 & 16 & 29 &  1 &  8 & 31 & 17 &  4 & 32 & 25 &  2 & 16 & 29 &  1 &  8 & 31 & 17 &  4 & 32 & 25 &  2 & 16 & 29 &  1 &  8 & 31 \\
    9   &    &  1 &  9 & 15 &  3 & 27 & 12 &  9 & 15 &  3 & 27 & 12 &  9 & 15 &  3 & 27 & 12 &  9 & 15 &  3 & 27 & 12 &  9 & 15 &  3 & 27 & 12 &  9 & 15 &  3 & 27 & 12 &  9 & 15 \\
   10   &    &  1 & 10 &  1 & 10 &  1 & 10 &  1 & 10 &  1 & 10 &  1 & 10 &  1 & 10 &  1 & 10 &  1 & 10 &  1 & 10 &  1 & 10 &  1 & 10 &  1 & 10 &  1 & 10 &  1 & 10 &  1 & 10 &  1 \\
   11   &    &  1 & 11 & 22 & 11 & 22 & 11 & 22 & 11 & 22 & 11 & 22 & 11 & 22 & 11 & 22 & 11 & 22 & 11 & 22 & 11 & 22 & 11 & 22 & 11 & 22 & 11 & 22 & 11 & 22 & 11 & 22 & 11 & 22 \\
   12   &    &  1 & 12 & 12 & 12 & 12 & 12 & 12 & 12 & 12 & 12 & 12 & 12 & 12 & 12 & 12 & 12 & 12 & 12 & 12 & 12 & 12 & 12 & 12 & 12 & 12 & 12 & 12 & 12 & 12 & 12 & 12 & 12 & 12 \\
   13   &    &  1 & 13 &  4 & 19 & 16 & 10 & 31 &  7 & 25 & 28 &  1 & 13 &  4 & 19 & 16 & 10 & 31 &  7 & 25 & 28 &  1 & 13 &  4 & 19 & 16 & 10 & 31 &  7 & 25 & 28 &  1 & 13 &  4 \\
   14   &    &  1 & 14 & 31 &  5 &  4 & 23 & 25 & 20 & 16 & 26 &  1 & 14 & 31 &  5 &  4 & 23 & 25 & 20 & 16 & 26 &  1 & 14 & 31 &  5 &  4 & 23 & 25 & 20 & 16 & 26 &  1 & 14 & 31 \\
   15   &    &  1 & 15 & 27 &  9 &  3 & 12 & 15 & 27 &  9 &  3 & 12 & 15 & 27 &  9 &  3 & 12 & 15 & 27 &  9 &  3 & 12 & 15 & 27 &  9 &  3 & 12 & 15 & 27 &  9 &  3 & 12 & 15 & 27 \\
   16   &    &  1 & 16 & 25 &  4 & 31 &  1 & 16 & 25 &  4 & 31 &  1 & 16 & 25 &  4 & 31 &  1 & 16 & 25 &  4 & 31 &  1 & 16 & 25 &  4 & 31 &  1 & 16 & 25 &  4 & 31 &  1 & 16 & 25 \\
   17   &    &  1 & 17 & 25 & 29 & 31 & 32 & 16 &  8 &  4 &  2 &  1 & 17 & 25 & 29 & 31 & 32 & 16 &  8 &  4 &  2 &  1 & 17 & 25 & 29 & 31 & 32 & 16 &  8 &  4 &  2 &  1 & 17 & 25 \\
   18   &    &  1 & 18 & 27 & 24 &  3 & 21 & 15 &  6 &  9 & 30 & 12 & 18 & 27 & 24 &  3 & 21 & 15 &  6 &  9 & 30 & 12 & 18 & 27 & 24 &  3 & 21 & 15 &  6 &  9 & 30 & 12 & 18 & 27 \\
   19   &    &  1 & 19 & 31 & 28 &  4 & 10 & 25 & 13 & 16 &  7 &  1 & 19 & 31 & 28 &  4 & 10 & 25 & 13 & 16 &  7 &  1 & 19 & 31 & 28 &  4 & 10 & 25 & 13 & 16 &  7 &  1 & 19 & 31 \\
   20   &    &  1 & 20 &  4 & 14 & 16 & 23 & 31 & 26 & 25 &  5 &  1 & 20 &  4 & 14 & 16 & 23 & 31 & 26 & 25 &  5 &  1 & 20 &  4 & 14 & 16 & 23 & 31 & 26 & 25 &  5 &  1 & 20 &  4 \\
   21   &    &  1 & 21 & 12 & 21 & 12 & 21 & 12 & 21 & 12 & 21 & 12 & 21 & 12 & 21 & 12 & 21 & 12 & 21 & 12 & 21 & 12 & 21 & 12 & 21 & 12 & 21 & 12 & 21 & 12 & 21 & 12 & 21 & 12 \\
   22   &    &  1 & 22 & 22 & 22 & 22 & 22 & 22 & 22 & 22 & 22 & 22 & 22 & 22 & 22 & 22 & 22 & 22 & 22 & 22 & 22 & 22 & 22 & 22 & 22 & 22 & 22 & 22 & 22 & 22 & 22 & 22 & 22 & 22 \\
   23   &    &  1 & 23 &  1 & 23 &  1 & 23 &  1 & 23 &  1 & 23 &  1 & 23 &  1 & 23 &  1 & 23 &  1 & 23 &  1 & 23 &  1 & 23 &  1 & 23 &  1 & 23 &  1 & 23 &  1 & 23 &  1 & 23 &  1 \\
   24   &    &  1 & 24 & 15 & 30 & 27 & 21 &  9 & 18 &  3 &  6 & 12 & 24 & 15 & 30 & 27 & 21 &  9 & 18 &  3 &  6 & 12 & 24 & 15 & 30 & 27 & 21 &  9 & 18 &  3 &  6 & 12 & 24 & 15 \\
   25   &    &  1 & 25 & 31 & 16 &  4 &  1 & 25 & 31 & 16 &  4 &  1 & 25 & 31 & 16 &  4 &  1 & 25 & 31 & 16 &  4 &  1 & 25 & 31 & 16 &  4 &  1 & 25 & 31 & 16 &  4 &  1 & 25 & 31 \\
   26   &    &  1 & 26 & 16 & 20 & 25 & 23 &  4 &  5 & 31 & 14 &  1 & 26 & 16 & 20 & 25 & 23 &  4 &  5 & 31 & 14 &  1 & 26 & 16 & 20 & 25 & 23 &  4 &  5 & 31 & 14 &  1 & 26 & 16 \\
   27   &    &  1 & 27 &  3 & 15 &  9 & 12 & 27 &  3 & 15 &  9 & 12 & 27 &  3 & 15 &  9 & 12 & 27 &  3 & 15 &  9 & 12 & 27 &  3 & 15 &  9 & 12 & 27 &  3 & 15 &  9 & 12 & 27 &  3 \\
   28   &    &  1 & 28 & 25 &  7 & 31 & 10 & 16 & 19 &  4 & 13 &  1 & 28 & 25 &  7 & 31 & 10 & 16 & 19 &  4 & 13 &  1 & 28 & 25 &  7 & 31 & 10 & 16 & 19 &  4 & 13 &  1 & 28 & 25 \\
   29   &    &  1 & 29 & 16 &  2 & 25 & 32 &  4 & 17 & 31 &  8 &  1 & 29 & 16 &  2 & 25 & 32 &  4 & 17 & 31 &  8 &  1 & 29 & 16 &  2 & 25 & 32 &  4 & 17 & 31 &  8 &  1 & 29 & 16 \\
   30   &    &  1 & 30 &  9 &  6 & 15 & 21 &  3 & 24 & 27 & 18 & 12 & 30 &  9 &  6 & 15 & 21 &  3 & 24 & 27 & 18 & 12 & 30 &  9 &  6 & 15 & 21 &  3 & 24 & 27 & 18 & 12 & 30 &  9 \\
   31   &    &  1 & 31 &  4 & 25 & 16 &  1 & 31 &  4 & 25 & 16 &  1 & 31 &  4 & 25 & 16 &  1 & 31 &  4 & 25 & 16 &  1 & 31 &  4 & 25 & 16 &  1 & 31 &  4 & 25 & 16 &  1 & 31 &  4 \\
   32   &    &  1 & 32 &  1 & 32 &  1 & 32 &  1 & 32 &  1 & 32 &  1 & 32 &  1 & 32 &  1 & 32 &  1 & 32 &  1 & 32 &  1 & 32 &  1 & 32 &  1 & 32 &  1 & 32 &  1 & 32 &  1 & 32 &  1 \\
\end{tabular}
\end{footnotesize}

  \caption{$m^a \quad (\mathcal{R}_{33})$}
  \label{modex-33}
\end{table}

%% Develop the equation that describes the pattern.
Notice that this table also contains identity columns.
They are columns 1, 11, 21, and 31.
Also notice that column 2 is the same as columns 12 and 22, 
column 3 is the same as columns 13 and 23, and so on.
In fact, the entire block of columns 1 through 10 is repeated in columns 11 through 20,
and this block pattern continues to repeat for columns beyond 20.

\newpage

We can easily represent this wallpaper pattern with an equation.
Any column $a$ is the same as column $10+a$ and column $20+a$ and so on.
We use the notation $k \cdot 10$ to denote some multiple of 10.
So for any $m \in Z_{33}$ and any integer $a > 0$ we have
\[  m^a = m^{k \cdot 10 + a}  \]

%% Define the Carmichael function value and give the equation.
Each row in Table \ref{modex-33} is a sequence of exponential products.
Each sequence is a 1 followed by a series of cycles.
These cycles have various periods.
For this table the periods are 1, 2, 5, and 10.
The period of the longest cycle is symbolized by $\lambda$.
This is also known as the \emph{Carmichael function value}.
For this particular exponential table we have $\lambda=10$.
However, for any two distinct primes $p$ and $q$,
it turns out that the Carmichael function value $\lambda$ 
is the \emph{least common multiple} of $p-1$ and $q-1$.
\[  \lambda = \lcm(p-1,q-1)  \]

%% Describe how to compute an lcm and give an example.
When $p$ and $q$ are small we can compute the Carmichael function value by hand.
For example, let $p=11$ and $q=13$ so that $\lambda=\lcm(10,12)$.
The multiples of 10 are 10, 20, 30, etc.
The multiples of 12 are 12, 24, 36, etc.
The multiples that are common to both are 60, 120, 180, etc.
The least of these is 60.
\begin{center}
\begin{tabular}{lcl}
  Multiples of 10 &=& 10, 20, 30, 40, 50, 60, 70, 80, 90, 100, 110, 120, \ldots \\
  Multiples of 12 &=& 12, 24, 36, 48, 60, 72, 84, 96, 108, 120, 132, \ldots \\
  Common to both &=& 60, 120, 180, \ldots \\
  $\lcm(10,12)$ &=& 60
\end{tabular}
\end{center}

%% Substitute the 10 with $\lambda$.
We described the wallpaper pattern of Table \ref{modex-33} 
using the equation $m^a=m^{k \cdot 10 + a}$.  We can replace the 10 with $\lambda$.
This gives us $m^a=m^{k\lambda + a}$.
This equation holds for primes $p=3$ and $q=11$.
But does it hold for \emph{any} pair of primes?
We assert that it does and we offer the following theorem without proof.

\paragraph{Wallpaper theorem:}
Given two distinct primes $p$ and $q$, the ring $\mathcal{R}_{pq}$,
the Carmichael function value $\lambda=\lcm(p-1,q-1)$, 
any $m \in Z_{pq}$, any integer $a > 0$, and any integer $k \ge 0$, then
\[  m^a = m^{k\lambda + a}  \]

RSA uses a special case of the Wallpaper theorem where $a=1$.
We call this special case the \emph{Carmichael identity}.
The $m^a$ on the left side is replaced by $m$, since $m^1=m$.
The $m^{k\lambda + a}$ on the right side is replaced by $m^{k\lambda + 1}$.

\paragraph{Carmichael identity:}
Given two distinct primes $p$ and $q$, the ring $\mathcal{R}_{pq}$,
the Carmichael function value $\lambda=\lcm(p-1,q-1)$, 
any $m \in Z_{pq}$, and any integer $k \ge 0$, then
\begin{equation} \label{eq:carm-id}
  m = m^{k\lambda + 1}
\end{equation}

