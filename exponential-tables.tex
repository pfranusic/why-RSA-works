%%%% why-RSA-works/exponential-tables.tex
%%%% Copyright 2012 Peter Franusic.
%%%% All rights reserved.
%%%%

%% Define an exponential product and give an example.
We now take a closer look at exponential products $m^a$ in the ring $\mathcal{R}_n$.
When $n$ is very small we can compute exponential products by hand.
As an example we compute $7^3$ in the ring $\mathcal{R}_{15}$ using Table \ref{otimes-15}.
\[  7^3 \quad = \quad 7 \otimes 7 \otimes 7 \quad = \quad (7 \otimes 7) \otimes 7 \quad 
= \quad 4 \otimes 7 \quad = \quad 13  \]

%% Define an exponential table and give an example.
We can go on to calculate the exponential product 
of each pair of elements in $Z_{15}$ and put them all in a table.
Table \ref{modex-15} specifies the exponential products $m^a$ in the ring $\mathcal{R}_{15}$.
The product of $7^3$ is at the intersection of row 7 and column 3.
\vspace{2ex}
%%%% modex table
\begin{table}[!h]
  \begin{center}
    %%%% why-RSA-works/modex-15.tex
%%%% Copyright 2012 Peter Franusic.
%%%% All rights reserved.
%%%%
\begin{footnotesize}
\begin{tabular}
    {c@{ }c@{ }c@{ }c@{ }c@{ }c@{ }c@{ }c@{ }c@{ }c@{ }c@{ }c@{ }c@{ }c@{ }c@{ }c@{ }c}
        & \phantom{X}
             &  0 &  1 &  2 &  3 &  4 &  5 &  6 &  7 &  8 &  9 & 10 & 11 & 12 & 13 & 14 \\
        &    & \phantom{10}  &    &    &    &    &    &    &    &    &    &    &    &    &    &    \\
    0   &    &  1 &  0 &  0 &  0 &  0 &  0 &  0 &  0 &  0 &  0 &  0 &  0 &  0 &  0 &  0 \\
    1   &    &  1 &  1 &  1 &  1 &  1 &  1 &  1 &  1 &  1 &  1 &  1 &  1 &  1 &  1 &  1 \\
    2   &    &  1 &  2 &  4 &  8 &  1 &  2 &  4 &  8 &  1 &  2 &  4 &  8 &  1 &  2 &  4 \\
    3   &    &  1 &  3 &  9 & 12 &  6 &  3 &  9 & 12 &  6 &  3 &  9 & 12 &  6 &  3 &  9 \\
    4   &    &  1 &  4 &  1 &  4 &  1 &  4 &  1 &  4 &  1 &  4 &  1 &  4 &  1 &  4 &  1 \\
    5   &    &  1 &  5 & 10 &  5 & 10 &  5 & 10 &  5 & 10 &  5 & 10 &  5 & 10 &  5 & 10 \\
    6   &    &  1 &  6 &  6 &  6 &  6 &  6 &  6 &  6 &  6 &  6 &  6 &  6 &  6 &  6 &  6 \\
    7   &    &  1 &  7 &  4 & 13 &  1 &  7 &  4 & 13 &  1 &  7 &  4 & 13 &  1 &  7 &  4 \\
    8   &    &  1 &  8 &  4 &  2 &  1 &  8 &  4 &  2 &  1 &  8 &  4 &  2 &  1 &  8 &  4 \\
    9   &    &  1 &  9 &  6 &  9 &  6 &  9 &  6 &  9 &  6 &  9 &  6 &  9 &  6 &  9 &  6 \\
   10   &    &  1 & 10 & 10 & 10 & 10 & 10 & 10 & 10 & 10 & 10 & 10 & 10 & 10 & 10 & 10 \\
   11   &    &  1 & 11 &  1 & 11 &  1 & 11 &  1 & 11 &  1 & 11 &  1 & 11 &  1 & 11 &  1 \\
   12   &    &  1 & 12 &  9 &  3 &  6 & 12 &  9 &  3 &  6 & 12 &  9 &  3 &  6 & 12 &  9 \\
   13   &    &  1 & 13 &  4 &  7 &  1 & 13 &  4 &  7 &  1 & 13 &  4 &  7 &  1 & 13 &  4 \\
   14   &    &  1 & 14 &  1 & 14 &  1 & 14 &  1 & 14 &  1 & 14 &  1 & 14 &  1 & 14 &  1 \\
\end{tabular}
\end{footnotesize}

    \caption{$m^a \quad (\mathcal{R}_{15})$}
    \label{modex-15}
  \end{center}
\end{table}

%% Define a cycle and point out examples in the table.
Now consider the product sequence in row 3 (shown below).
Notice how the sequence starts at 1 and then repeats itself.
The shortest repetitive part of a sequence is called a \emph{cycle}.
The cycle in row 3 is (3, 9, 12, 6).
The \emph{period} of this cycle is 4.
\[ 1 \quad \overbrace{3 \quad 9 \quad 12 \quad 6}
     \quad \overbrace{3 \quad 9 \quad 12 \quad 6} \quad \cdots \]

%% Define an identity column and point out examples in the table.
Each row in Table \ref{modex-15} is a sequence that starts with 1 followed by a series of cycles.
Each cycle in the various rows has a period of either 1 or 2 or 4.
Remarkably, all of the cycles line up vertically in such a way
as to provide what may be called \emph{identity columns}.
Consider columns 1, 5, 9, and 13.  These are the identity columns in the table.
Each is identical to the row number column on the left side of the table.
So for any $m \in Z_{15}$ we have
\[  m^1 =  m^5  =  m^9  =  m^{13}  \]

