%%%% why-RSA-works/references.tex
%%%% Copyright 2012 Peter Franusic.
%%%% All rights reserved.
%%%%

\begin{thebibliography}{99}

\bibitem{RSA-paper}
  R. L. Rivest, A. Shamir, and L. Adleman.
  A method for obtaining digital signatures and public-key cryptosystems.
  \emph{Communications of the ACM}, 21(2):120-126.

\bibitem{ray-attack}
  Andreas de Vries.  The ray attack, an inefficient trial to break RSA cryptosystems.
  FH S\"udwestfalen University of Applied Sciences, Haldener StraBe 182, D-58095 Hagen, 2003.

\bibitem{wiki-Rings}
  Ring (mathematics). \emph{Wikipedia}, \verb$www.wikipedia.com$.

\bibitem{Koc}
  \c Cetin Kaya Ko\c c. High-speed RSA implementation. RSA Labs, 1994.

\bibitem{Schneier}
  Bruce Schneier. \emph{Applied Cryptography}. John Wiley \& Sons, 1994.

\bibitem{HAC}
  A. Menezes, P. van Oorschot, and S. Vanstone.
  \emph{Handbook of Applied Cryptography}. CRC Press, 1996. 

\bibitem{RSA-standard}
  PKCS \#1 v2.1: RSA cryptography standard. RSA Labs, 2002.

\bibitem{Shor}
  Peter W. Shor,
  ``Polynomial-Time Algorithms for Prime Factorization and 
  Discrete Logarithms on a Quantum Computer,'' 1996.

\bibitem{RSA-768}
  Thorsten Kleinjung et al.
  Factorization of a 768-bit RSA modulus.
  Version 1.4, February 18, 2010.

\bibitem{RSA-problem}
  Ronald L. Rivest and Burt Kaliski. RSA problem.  RSA Labs, 2003.

\end{thebibliography}

