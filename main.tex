%%%% why-RSA-works/main.tex
%%%% Copyright 2012 Peter Franusic.
%%%% All rights reserved.
%%%%

\documentclass{article}
%\pagestyle{empty}

%%%% Various environments
\usepackage{verbatim}
\usepackage{graphicx}
\usepackage{latexsym}
\usepackage{amssymb}

%%%% Easy-vision mode
\usepackage[usenames]{color}
% \pagecolor{black}
% \color{green}

%%%% math-mode commands
\newcommand{\lcm}{\mathrm{lcm}}

%%%% PDF metadata
\pdfinfo
{ /Title (Why RSA Works)
  /Author (Peter Franusic)
  /Subject (Rivest-Shamir-Adleman algorithm)
  /Keywords (RSA, Carmichael, public-key, cryptography)
}

%%%% European-style paragraphs
%%%% IMPORTANT: \begin{document} must follow for this to work.
\setlength{\parindent}{0pt} 
\setlength{\parskip}{1.3ex} 

%%%% Title block
\title{\textbf{\huge{Why RSA Works}}}
\author{Peter Franu\v si\'c
  \footnote{
    Copyright 2012 Peter Franu\v si\'c.
    All rights reserved.
    Email: \texttt{pete@sargo.com}}}
\date{}

\begin{document}

%% Title page
\maketitle
\thispagestyle{empty}
\vspace{8ex}
\input{intro.tex}

\newpage
\section{Huge integers}
\input{huge-integers.tex}

\newpage
\section{Simulation}
\input{simulation.tex}

\newpage
\section{Rings}
\input{rings.tex}

\section{The set $Z_n$}
\input{set-Zn.tex}

\newpage
\section{The $\oplus$ operator}
\input{oplus-operator.tex}

\newpage
\section{The $\otimes$ operator}
\input{otimes-operator.tex}

\newpage
\section{Exponential notation}
\input{exponential-notation.tex}

\newpage
\section{The modex function}
\input{modex-function.tex}

\newpage
\section{Exponent arithmetic}
\input{exponent-arithmetic.tex}

\newpage
\section{Multiple-plus-one}
\input{multiple-plus-one.tex}

\newpage
\section{Exponential tables}
\input{exponential-tables.tex}

\newpage
\section{Wallpaper}
\input{wallpaper.tex}

\newpage
\section{Mappings}
\input{mappings.tex}

\newpage
\section{A simple proof}
\input{simple-proof.tex}

\newpage
\section{Hard problems}
\input{hard-problems.tex}

\newpage
\section{Conclusions}
\input{conclusions.tex}

\newpage
%%%% References
\input{references.tex}

\end{document}

