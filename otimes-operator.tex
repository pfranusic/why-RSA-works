%%%% why-RSA-works/otimes-operator.tex
%%%% Copyright 2012 Peter Franusic.
%%%% All rights reserved.
%%%%

When $n$ is small, the $\otimes$ operator can be specified using a table.
Table \ref{otimes-15} specifies the $\otimes$ operator for the ring $\mathcal{R}_{15}$.
The format is the same as Table \ref{oplus-15}.  The values, of course, are different.
Note the rose-like pattern visible in the table.
The table is symmetrical about the diagonals.
If we ignore column 0, then
row 1 is the reverse of row 14, row 2 is the reverse of row 13, etc.

\vspace{2ex}
%%%% otimes-15 table
\begin{table}[!ht]
  \begin{center}
    %%%% why-RSA-works/otimes-15.tex
%%%% Copyright 2012 Peter Franusic.
%%%% All rights reserved.
%%%%
\begin{footnotesize}
\begin{tabular}
    {c@{ }c@{ }c@{ }c@{ }c@{ }c@{ }c@{ }c@{ }c@{ }c@{ }c@{ }c@{ }c@{ }c@{ }c@{ }c@{ }c}
        & \phantom{X}
             &  0 &  1 &  2 &  3 &  4 &  5 &  6 &  7 &  8 &  9 & 10 & 11 & 12 & 13 & 14 \\
        &    &    &    &    &    &    &    &    &    &    &    &    &    &    &    &    \\
    0   &    &  0 &  0 &  0 &  0 &  0 &  0 &  0 &  0 &  0 &  0 &  0 &  0 &  0 &  0 &  0 \\
    1   &    &  0 &  1 &  2 &  3 &  4 &  5 &  6 &  7 &  8 &  9 & 10 & 11 & 12 & 13 & 14 \\
    2   &    &  0 &  2 &  4 &  6 &  8 & 10 & 12 & 14 &  1 &  3 &  5 &  7 &  9 & 11 & 13 \\
    3   &    &  0 &  3 &  6 &  9 & 12 &  0 &  3 &  6 &  9 & 12 &  0 &  3 &  6 &  9 & 12 \\
    4   &    &  0 &  4 &  8 & 12 &  1 &  5 &  9 & 13 &  2 &  6 & 10 & 14 &  3 &  7 & 11 \\
    5   &    &  0 &  5 & 10 &  0 &  5 & 10 &  0 &  5 & 10 &  0 &  5 & 10 &  0 &  5 & 10 \\
    6   &    &  0 &  6 & 12 &  3 &  9 &  0 &  6 & 12 &  3 &  9 &  0 &  6 & 12 &  3 &  9 \\
    7   &    &  0 &  7 & 14 &  6 & 13 &  5 & 12 &  4 & 11 &  3 & 10 &  2 &  9 &  1 &  8 \\
    8   &    &  0 &  8 &  1 &  9 &  2 & 10 &  3 & 11 &  4 & 12 &  5 & 13 &  6 & 14 &  7 \\
    9   &    &  0 &  9 &  3 & 12 &  6 &  0 &  9 &  3 & 12 &  6 &  0 &  9 &  3 & 12 &  6 \\
   10   &    &  0 & 10 &  5 &  0 & 10 &  5 &  0 & 10 &  5 &  0 & 10 &  5 &  0 & 10 &  5 \\
   11   &    &  0 & 11 &  7 &  3 & 14 & 10 &  6 &  2 & 13 &  9 &  5 &  1 & 12 &  8 &  4 \\
   12   &    &  0 & 12 &  9 &  6 &  3 &  0 & 12 &  9 &  6 &  3 &  0 & 12 &  9 &  6 &  3 \\
   13   &    &  0 & 13 & 11 &  9 &  7 &  5 &  3 &  1 & 14 & 12 & 10 &  8 &  6 &  4 &  2 \\
   14   &    &  0 & 14 & 13 & 12 & 11 & 10 &  9 &  8 &  7 &  6 &  5 &  4 &  3 &  2 &  1 \\
\end{tabular}
\end{footnotesize}

    \caption{$a \otimes b \quad (\mathcal{R}_{15})$}
    \label{otimes-15}
  \end{center}
\end{table}

Table \ref{otimes-15} specifies the value of $a \otimes b$ for every possible pair of $a$ and $b$.
For example, let $a=11$ and $b=8$.
The value of $11 \otimes 8$ is specified at the intersection of row $11$ and column $8$.
This value is $13$.  Therefore $11 \otimes 8 = 13$.

Notice that every element in the table is in the set $Z_{15}$.
This demonstrates the \emph{multiplicative closure} property of rings.
The multiplicative closure property states that for every pair of elements $a$ and $b$ in $Z_n$,
the product $a \otimes b$ is also an element in $Z_n$.
\[ a \otimes b \in Z_n \]

The value of $a \otimes b$ can also be specified using a rule.
To compute $11 \otimes 8$ we first calculate $11 \times 8$ to get 88.
Since 88 is not an element in $Z_{15}$ we subtract a multiple of the modulus, $kn$.
In this case, $kn = 5 \times 15 = 75$.  Therefore $88 - 75 = 13$.
And $13 \in Z_{15}$ so 13 is our final result.

In general, the $\otimes$ operator takes two integers $a$ and $b$, 
multiplies them together using normal multiplication, 
then subtracts some multiple of $n$ such that the final value is in $Z_n$.
In other words, we subtract whichever $kn$ works in order to get closure, 
where $a \otimes b \in Z_n$.
\[  a \otimes b = ab - kn  \]

