%%%% why-RSA-works/rings.tex
%%%% Copyright 2012 Peter Franusic.
%%%% All rights reserved.
%%%%

RSA uses mathematical structures called rings.
A \emph{ring} is a set equipped with two binary operators.\cite{wiki-Rings}
The ring displays several well-defined algebraic properties,
including both additive closure and multiplicative closure.

Recall that a set is simply a collection of elements.
These elements can be anything, but in the case of RSA, the elements are integers.
RSA uses sets with a finite number of elements.
The number of elements in a set is called the \emph{modulus}.
The modulus is represented by the symbol $n$.

A binary operator is something that takes two elements and computes a third.
Rings use two binary operators, which we denote here as
$\oplus$ (pronounced \textsf{OH plus}) and $\otimes$ (pronounced \textsf{OH times}).
The $\oplus$ operator is similar to addition.
The $\otimes$ operator is similar to multiplication.

In general, we say that the ring  $\mathcal{R}_n$
consists of the set $Z_n$, the $\oplus$ operator, and the $\otimes$ operator.
\[  \mathcal{R}_n = (Z_n,\oplus,\otimes)  \]

